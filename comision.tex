\documentclass[spanish,11pt,a4paper]{article}
\usepackage[T1]{fontenc}
\usepackage[width=20cm, height=26cm]{geometry}
\usepackage{graphicx}
\usepackage{mathtools}
\usepackage{amssymb}
\usepackage{amsthm}
\usepackage{xcolor}
\usepackage{babel}
\usepackage{hyperref}
\usepackage{longtable}
\title{Informe comisión y promoción}
\author{Brayan Romero}
\begin{document}
	
	\centering
	
	\begin{tabular}{|p{3cm}|p{12cm}|p{3cm}|}
		\hline
		\vspace{0.05cm}
		\centering
		\includegraphics[scale=0.35]{logo.png}
		&
		\begin{center}
			\textbf{GIMNASIO PSICOPEDAGÓGICO SUBA}
			
			\vspace{0.1cm}
			El amor, fuente fundamental para la formación integral
			
			\vspace{0.1cm}
			\textbf{COMISIÓN DE EVALUACIÓN}
			
			\vspace{0.1cm}
			\textbf{INFORME GENERAL E INDIVIDUAL}
		\end{center}&
		\vspace{1.25cm}
		FR - GA - 021 V1\\
		\hline
	\end{tabular}
	
	\vspace{0.4cm}
	
	\begin{tabular}{|c c|c c|c c|c c|}
		\hline
		\textbf{Director:} & B. Romero & \textbf{Curso:} & 10 A & \textbf{Fecha:} & \today & \textbf{Trimestre:} & Segundo \\
		\hline
	\end{tabular}
	\vspace{0.4cm}
	
	\textbf{INFORME GENERAL DEL GRUPO}
	
	\vspace{0.4cm}
	
	\resizebox{\textwidth}{!}{
		\begin{tabular}{|c|p{4cm}|p{4cm}|p{4cm}|p{4cm}|}
			\hline
			\textbf{GRADO} & \textbf{ASPECTO DISCIPLINARIO} & \textbf{ASPECTO ACADÉMICO} & \textbf{¿QUÉ HA HECHO?} & \textbf{¿QUÉ PROPONE HACER?} \\
			\hline \hline
			
			
			10 A &
			%ASPECTO DISCIPLINARIO
			Grado décimo A es un curso compuesto por 21 estudiantes, 11 niñas y 10 niños, tiene 2 niños de inclusión y un estudiante es nuevo.
			
			Es un curso que se esfuerza por seguir las normas de los docentes, y trata de no verse involucrado en cualquier tipo de faltas.
			
			Los estudiantes de grado décimo manejan con regularidad un lenguaje soez entre sus propios compañeros, este comportamiento ha disminuido por los continuos llamados de atención y el acompañamiento de los padres de familia desde casa.
			
			Los estudiantes de grado décimo muestran ser un grupo unido que se apoya constantemente, y busca la forma de que entre todos puedan llevarse bien. 
			
			Grado décimo es un curso muy sensible, algunas situaciones del día a día los afecta demasiado, ocasionando cargas extra en los estudiantes.& 
			
			%ASPECTO ACADÉMICO
			Grado décimo, en el aspecto académico es un buen curso, se esmera en entregar sus tareas y trabajos a tiempo. Les gusta realizar preguntas en clase, ayudar a sus compañeros cuando algún tema se les complica. 
			
			Los estudiantes de grado décimo son muy atentos a las indicaciones de los docentes, les gusta apoyar a los profesores en todo tipo de actividades.
			
			Es un grupo muy artístico, les encanta dibujar, y hacer carteleras, de las distintas temáticas de las clases.  
			
			Cuatro estudiantes pierden una asignatura, Isabella Acevedo pierde trigonometría, Lorieth Chavarro pierde trigonometría, Vanessa Ortiz pierde trigonometría y Samuel Quintero pierde Microcomunidad& 
			
			%¿QUÉ HA HECHO?
			Se ha hecho seguimiento permanente por medio del registro de actividades académicas a diario, verificando y firmado desde casa por medio de los padres de familia.
			
			Seguimiento de los estudiantes, por agenda y plataforma, citación de padres de familia cuando es necesario, y la falta del estudiante es recurrente.
			
			Se han realizado distintas actividades para la mejor convivencia en el aula, así como la continua concienciación acerca de la importancia de seguir instrucciones y acatar las normas de clase. &
			
			%¿QUÉ PROPONE HACER? 
			Propongo  que para el próximo trimestre académico, los estudiantes de grado décimo A, se esfuercen más en las asignaturas que se encuentran en bajo y en básico, y en aquellas asignaturas que presentaros dificultades académicas.
			
			Seguir fortaleciendo las debilidades en el área de matemáticas y física, para tener avances más significativos y puedan superar todos los desempeños bajos.
			
			Se propone seguir enfatizando en los estudiantes, el buen trato y las buenas costumbres, seguir haciendo concienciación a los estudiantes sobre el uso del lenguaje vulgar y los malos gestos, ya que estos conllevan una falta de respeto.
			\\
			\hline
	\end{tabular}}
	\newpage
	
	\textbf{INFORME INDIVIDUAL DE CADA ESTUDIANTE}
	
	\begin{longtable}{|p{3.5cm}|p{3.5cm}|p{3.5cm}|p{3.5cm}|p{3.5cm}|}
		\hline
		\textbf{ESTUDIANTE} & \textbf{ASPECTO DISCIPLINARIO} & \textbf{ASPECTO ACADÉMICO} & \textbf{¿QUÉ HA HECHO?} & \textbf{¿QUÉ PROPONE HACER?} \\
		\hline \hline
		\endfirsthead
		
		\hline
		\textbf{ESTUDIANTE} & \textbf{ASPECTO DISCIPLINARIO} & \textbf{ASPECTO ACADÉMICO} & \textbf{¿QUÉ HA HECHO?} & \textbf{¿QUÉ PROPONE HACER?} \\
		\hline \hline
		\endhead
		
		%Estudiante
		Acevedo Vanegas Isabella & 
		%Disciplinario
		Es una estudiante que sigue las indicaciones del docente, evita cometer faltas de cualquier tipo, presenta una buena conducta en el aula de clases. & 
		%Académico
		La estudiante pierde trigonometría. 
		La estudiante tiene con pérdida: 2 desempeños de Actitudinal De Padres, 1 desempeños de Español, 1 desempeños de Física y 5 desempeños de Trigonometría& 
		%¿Qué ha hecho?
		Seguimiento continúo académico y convivencial de la estudiante, seguimiento de huellas por plataforma y agenda. Diálogo constante con la estudiante, aconsejando y acompañando todos los días. & 
		%¿Qué propone hacer?
		Continuar con el seguimiento constante, seguir concientizando a los estudiantes sobre el cumplimiento de las normas del colegio, fomentar aún más el compañerismo, seguir recalcando en los estudiantes las buenas costumbres y el buen trato entre compañeros y docentes. Seguir motivando a ser mejores personas día con día.\\
		\hline
		
		%__________________________
		%Estudiante
		Chavarro Reyes Lorieth Sofia & 
		%Disciplinario
		Es una estudiante que sigue las indicaciones del docente, evita cometer faltas de cualquier tipo, presenta una buena conducta en el aula de clases. Colabora constantemente a los docentes. & 
		%Académico
		La estudiante pierde trigonometría.
		La estudiante tiene con pérdida: 1 desempeño de Actitudinal de Padres, 1 desempeño de Reading and Writing, 2 desempeños de Química, 2 desempeños de Ciencias Políticas y 6 desempeños de Trigonometría.& 
		%¿Qué ha hecho?
		Seguimiento continúo académico y convivencial de la estudiante, seguimiento de huellas por plataforma y agenda. Diálogo constante con la estudiante, aconsejando y acompañando todos los días. & 
		%¿Qué propone hacer?
		Continuar con el seguimiento constante, seguir concientizando a los estudiantes sobre el cumplimiento de las normas del colegio, fomentar aún más el compañerismo, seguir recalcando en los estudiantes las buenas costumbres y el buen trato entre compañeros y docentes. Seguir motivando a ser mejores personas día con día.\\
		\hline
		
		%__________________________
		%Estudiante
		Duitama Pérez Juan Pablo & 
		%Disciplinario
		Es un estudiante que sigue las indicaciones del docente, evita cometer faltas de cualquier tipo, presenta una buena conducta en el aula de clases. Es un estudiante crítico, sabe reconocer sus errores. & 
		%Académico
		El estudiante tiene con pérdida: 2 desempeños de Actitudinal de Padres y 1 desempeño de Trigonometría. & 
		%¿Qué ha hecho?
		Seguimiento continúo académico y convivencial del estudiante, seguimiento de huellas por plataforma y agenda. Diálogo constante con la estudiante, aconsejando y acompañando todos los días. & 
		%¿Qué propone hacer?
		Continuar con el seguimiento constante, seguir concientizando a los estudiantes sobre el cumplimiento de las normas del colegio, fomentar aún más el compañerismo, seguir recalcando en los estudiantes las buenas costumbres y el buen trato entre compañeros y docentes. Seguir motivando a ser mejores personas día con día.\\
		\hline
		
		%__________________________
		%Estudiante
		Garzon Toro Joustin Andres & 
		%Disciplinario
		Es un estudiante donde se han hecho reiterados llamados de atención por dormir en clases. & 
		%Académico
		El estudiante tiene con pérdida: 1 desempeño de Español, 1 desempeño Artes y 1 desempeño Plan Lector.  & 
		%¿Qué ha hecho?
		Seguimiento continúo académico y convivencial del estudiante, seguimiento de huellas por plataforma y agenda. Diálogo constante con la estudiante, aconsejando y acompañando todos los días. & 
		%¿Qué propone hacer?
		Continuar con el seguimiento constante, seguir concientizando a los estudiantes sobre el cumplimiento de las normas del colegio, fomentar aún más el compañerismo, seguir recalcando en los estudiantes las buenas costumbres y el buen trato entre compañeros y docentes. Seguir motivando a ser mejores personas día con día.\\
		\hline
		
		%__________________________
		%Estudiante
		Gonzalez Arguello Mariet Juliana & 
		%Disciplinario
		Es una estudiante que sigue las indicaciones del docente, evita cometer faltas de cualquier tipo, presenta una buena conducta en el aula de clases. & 
		%Académico
		La estudiante tiene con pérdida: 1 desempeño de Microcomunidad, 1 desempeño de Actitudinal de Padres y 2 desempeños de Trigonometría. & 
		%¿Qué ha hecho?
		Seguimiento continúo académico y convivencial de la estudiante, seguimiento de huellas por plataforma y agenda. Diálogo constante con la estudiante, aconsejando y acompañando todos los días. & 
		%¿Qué propone hacer?
		Continuar con el seguimiento constante, seguir concientizando a los estudiantes sobre el cumplimiento de las normas del colegio, fomentar aún más el compañerismo, seguir recalcando en los estudiantes las buenas costumbres y el buen trato entre compañeros y docentes. Seguir motivando a ser mejores personas día con día.\\
		\hline
		
		%__________________________
		%Estudiante
		Higuera Herrera Nicole Dallan & 
		%Disciplinario
		Es una estudiante que sigue las indicaciones del docente, evita cometer faltas de cualquier tipo, presenta una buena conducta en el aula de clases, le cuesta inciciar y terminar actividades, a menudo la estudiante no quiere realizar las actividades propuestas por el docente por estar copiando la información que se presenta en el tablero. & 
		%Académico
		La estudiante pierde SENA.
		La estudiante tiene con pérdida: 1 desempeño de Microcomunidad, 1 desempeño de Actitudinal de Padres, 2 desempeños de Física y 3 desempeños de SENA. & 
		%¿Qué ha hecho?
		Seguimiento continúo académico y convivencial de la estudiante, seguimiento de huellas por plataforma y agenda. Diálogo constante con la estudiante, aconsejando y acompañando todos los días. & 
		%¿Qué propone hacer?
		Continuar con el seguimiento constante, seguir concientizando a los estudiantes sobre el cumplimiento de las normas del colegio, fomentar aún más el compañerismo, seguir recalcando en los estudiantes las buenas costumbres y el buen trato entre compañeros y docentes. Seguir motivando a ser mejores personas día con día.\\
		\hline
		
		%__________________________
		%Estudiante
		Jimenez Vanegas Juan Diego & 
		%Disciplinario
		Es un estudiante  que toca hacer llamados de atención constantemente por interrumpir la clase, al estar hablando con sus compañeros. Es un estudiante, que es agresivo, responde mal a estudiantes, e incluso algunas veces a los docentes, necesita acompañamiento de psicología por el manejo de emociones.& 
		%Académico
		El estudiante tiene con pérdida: 1 desempeño de Microcomunidad, 2 desempeños de Actitudinal de Padres, 1 desempeño de Listening and Speaking, 1 desempeño de Reading and Writing, 1 desempeño de Química, 2 desempeños de Física, 1 desempeño de Artes y 2 desempeños de Trigonometría. & 
		%¿Qué ha hecho?
		Seguimiento continúo académico y convivencial del estudiante, seguimiento de huellas por plataforma y agenda. Diálogo constante con la estudiante, aconsejando y acompañando todos los días.  Citación a acudientes.& 
		%¿Qué propone hacer?
		Continuar con el seguimiento constante, seguir concientizando a los estudiantes sobre el cumplimiento de las normas del colegio, fomentar aún más el compañerismo, seguir recalcando en los estudiantes las buenas costumbres y el buen trato entre compañeros y docentes. Seguir motivando a ser mejores personas día con día.\\
		\hline
		
		%__________________________
		%Estudiante
		Lara Gallardo Juan Sebastián & 
		%Disciplinario
		Es un estudiante que sigue las indicaciones del docente, evita cometer faltas de cualquier tipo, presenta una buena conducta en el aula de clases. & 
		%Académico
		El estudiante no cuenta con ningún desempeño perdido, pasa en limpio el primer trimestre académico. & 
		%¿Qué ha hecho?
		Seguimiento continúo académico y convivencial del estudiante, seguimiento de huellas por plataforma y agenda. Diálogo constante con la estudiante, aconsejando y acompañando todos los días. & 
		%¿Qué propone hacer?
		Continuar con el seguimiento constante, seguir concientizando a los estudiantes sobre el cumplimiento de las normas del colegio, fomentar aún más el compañerismo, seguir recalcando en los estudiantes las buenas costumbres y el buen trato entre compañeros y docentes. Seguir motivando a ser mejores personas día con día.\\
		\hline
		
		%__________________________
		%Estudiante
		Matiz Camargo Angel Santiago & 
		%Disciplinario
		El estudiante debe firmar compromiso por retardos.
		Es un estudiante que a principio de año se veía metido en llamados de atención constante, tiene seguimiento desde coordinación con compromiso disciplinario por las reiteradas faltas, y llamados de atención. El estudiante ha mejorado progresivamente estas últimas semanas.  
		& 
		%Académico
		El estudiante tiene con pérdida: 1 desempeño de Microcomunidad, 1 desempeño de Actitudinal de Padres, 1 desempeño de Español, 3 desempeños de Química, 1 desempeño de Ciencias Políticas y 2 desempeños de Trigonometría. & 
		%¿Qué ha hecho?
		Seguimiento continúo académico y convivencial del estudiante, seguimiento de huellas por plataforma y agenda. Diálogo constante con la estudiante, aconsejando y acompañando todos los días. Citación a acudientes.& 
		%¿Qué propone hacer?
		Continuar con el seguimiento constante, seguir concientizando a los estudiantes sobre el cumplimiento de las normas del colegio, fomentar aún más el compañerismo, seguir recalcando en los estudiantes las buenas costumbres y el buen trato entre compañeros y docentes. Seguir motivando a ser mejores personas día con día.\\
		\hline
		
		%__________________________
		%Estudiante
		Mejia Obando Johan David & 
		%Disciplinario
		Es un estudiante que sigue las indicaciones del docente, evita cometer faltas de cualquier tipo, presenta una buena conducta en el aula de clases. & 
		%Académico
		El estudiante tiene con pérdida: 1 desempeño de Actitudinal de Padres, 1 desempeño de Reading and Writing, 1 desempeño de Español, 1 desempeño de Química y 1 desempeño de Trigonometría. & 
		%¿Qué ha hecho?
		Seguimiento continúo académico y convivencial del estudiante, seguimiento de huellas por plataforma y agenda. Diálogo constante con la estudiante, aconsejando y acompañando todos los días. & 
		%¿Qué propone hacer?
		Continuar con el seguimiento constante, seguir concientizando a los estudiantes sobre el cumplimiento de las normas del colegio, fomentar aún más el compañerismo, seguir recalcando en los estudiantes las buenas costumbres y el buen trato entre compañeros y docentes. Seguir motivando a ser mejores personas día con día.\\
		\hline
		
		%__________________________
		%Estudiante
		Melo Monroy Fredy Alejandro & 
		%Disciplinario
		Es un estudiante que sigue las indicaciones del docente, evita cometer faltas de cualquier tipo, presenta una buena conducta en el aula de clases. & 
		%Académico
		El estudiante tiene con pérdida: 2 desempeños de Actitudinal de Padres, 1 desempeño de Español y 2 desempeños de Trigonometría & 
		%¿Qué ha hecho?
		Seguimiento continúo académico y convivencial del estudiante, seguimiento de huellas por plataforma y agenda. Diálogo constante con la estudiante, aconsejando y acompañando todos los días. & 
		%¿Qué propone hacer?
		Continuar con el seguimiento constante, seguir concientizando a los estudiantes sobre el cumplimiento de las normas del colegio, fomentar aún más el compañerismo, seguir recalcando en los estudiantes las buenas costumbres y el buen trato entre compañeros y docentes. Seguir motivando a ser mejores personas día con día.\\
		\hline
		
		%__________________________
		%Estudiante
		Moreno Buitrago Samuel & 
		%Disciplinario
		Es un estudiante que se le hacen reiterados llamados de atención, por su comportamiento en clase y el trato con sus compañeros, uso de lenguaje soez para referirse a sus pares, el estudiante tiene compromiso por comportamiento. & 
		%Académico
		El estudiante tiene con pérdida: 2 desempeños de Actitudinal de Padres y 1 desempeño de Español & 
		%¿Qué ha hecho?
		Seguimiento continúo académico y convivencial del estudiante, seguimiento de huellas por plataforma y agenda. Diálogo constante con la estudiante, aconsejando y acompañando todos los días. Citación a acudientes. & 
		%¿Qué propone hacer?
		Continuar con el seguimiento constante, seguir concientizando a los estudiantes sobre el cumplimiento de las normas del colegio, fomentar aún más el compañerismo, seguir recalcando en los estudiantes las buenas costumbres y el buen trato entre compañeros y docentes. Seguir motivando a ser mejores personas día con día.\\
		\hline
		
		%__________________________
		%Estudiante
		Naranjo Moreno Dana Alejandra & 
		%Disciplinario
		Es una estudiante que sigue las indicaciones del docente, evita cometer faltas de cualquier tipo, presenta una buena conducta en el aula de clases. & 
		%Académico
		La estudiante tiene con pérdida:  1 desempeño de Microcomunidad, 1 desempeño de Actitudinal de Padres, 1 desempeño de Reading and Writing, 1 desempeño de Física y 3 desempeños de Trigonometría. & 
		%¿Qué ha hecho?
		Seguimiento continúo académico y convivencial de la estudiante, seguimiento de huellas por plataforma y agenda. Diálogo constante con la estudiante, aconsejando y acompañando todos los días. & 
		%¿Qué propone hacer?
		Continuar con el seguimiento constante, seguir concientizando a los estudiantes sobre el cumplimiento de las normas del colegio, fomentar aún más el compañerismo, seguir recalcando en los estudiantes las buenas costumbres y el buen trato entre compañeros y docentes. Seguir motivando a ser mejores personas día con día.\\
		\hline
		
		%__________________________
		%Estudiante
		Ortiz España Michelle Vanessa & 
		%Disciplinario
		A la estudiante se le hacen llamados de atención con regularidad por su relación de pareja con otro estudiante, constantemente. Es una estudiante participativa, le gusta colaborar en clase. & 
		%Académico
		La estudiante pierde trigonometría. 
		La estudiante tiene con pérdida: 1 desempeño de Microcomunidad, 2 desempeños de Actitudinal de Padres, 1 desempeño de Reading and Writing, 2 desempeños de Química, 1 desempeño de Física y 5 desempeños de Trigonometría.& 
		%¿Qué ha hecho?
		Seguimiento continúo académico y convivencial de la estudiante, seguimiento de huellas por plataforma y agenda. Diálogo constante con la estudiante, aconsejando y acompañando todos los días. & 
		%¿Qué propone hacer?
		Continuar con el seguimiento constante, seguir concientizando a los estudiantes sobre el cumplimiento de las normas del colegio, fomentar aún más el compañerismo, seguir recalcando en los estudiantes las buenas costumbres y el buen trato entre compañeros y docentes. Seguir motivando a ser mejores personas día con día.\\
		\hline
		
		%__________________________
		%Estudiante
		Páez Terreros Laura Valentina & 
		%Disciplinario
		Es una estudiante que sigue las indicaciones del docente, evita cometer faltas de cualquier tipo, presenta una buena conducta en el aula de clases. & 
		%Académico
		La estudiante tiene con pérdida: 1 desempeño de Actitudinal de Padres. & 
		%¿Qué ha hecho?
		Seguimiento continúo académico y convivencial de la estudiante, seguimiento de huellas por plataforma y agenda. Diálogo constante con la estudiante, aconsejando y acompañando todos los días. & 
		%¿Qué propone hacer?
		Continuar con el seguimiento constante, seguir concientizando a los estudiantes sobre el cumplimiento de las normas del colegio, fomentar aún más el compañerismo, seguir recalcando en los estudiantes las buenas costumbres y el buen trato entre compañeros y docentes. Seguir motivando a ser mejores personas día con día.\\
		\hline
		
		%__________________________
		%Estudiante
		Perez Garcia Maria Valentina & 
		%Disciplinario
		Es una estudiante que sigue las instrucciones de los docentes en el aula de clase, evita cometer cualquier tipo de faltas.  & 
		%Académico
		La estudiante tiene con pérdida: 1 desempeño de Microcomunidad, 1 desempeño de Actitudinal de Padres, 3 desempeños de Reading and Writing, 2 desempeños de Física y 3 desempeños de Trigonometría. & 
		%¿Qué ha hecho?
		Seguimiento continúo académico y convivencial de la estudiante, seguimiento de huellas por plataforma y agenda. Diálogo constante con la estudiante, aconsejando y acompañando todos los días. & 
		%¿Qué propone hacer?
		Continuar con el seguimiento constante, seguir concientizando a los estudiantes sobre el cumplimiento de las normas del colegio, fomentar aún más el compañerismo, seguir recalcando en los estudiantes las buenas costumbres y el buen trato entre compañeros y docentes. Seguir motivando a ser mejores personas día con día.\\
		\hline
		
		%__________________________
		%Estudiante
		Perez Cardenas Ibeth Tatiana & 
		%Disciplinario
		Es una estudiante que sigue las indicaciones del docente, evita cometer faltas de cualquier tipo, presenta una buena conducta en el aula de clases. & 
		%Académico
		La estudiante tiene con pérdida: 1 desempeño de Actitudinal de Padres, 1 desempeño de Química, 1 desempeño de Física y 1 desempeño de Trigonometría.	 & 
		%¿Qué ha hecho?
		Seguimiento continúo académico y convivencial de la estudiante, seguimiento de huellas por plataforma y agenda. Diálogo constante con la estudiante, aconsejando y acompañando todos los días. & 
		%¿Qué propone hacer?
		Continuar con el seguimiento constante, seguir concientizando a los estudiantes sobre el cumplimiento de las normas del colegio, fomentar aún más el compañerismo, seguir recalcando en los estudiantes las buenas costumbres y el buen trato entre compañeros y docentes. Seguir motivando a ser mejores personas día con día.\\
		\hline
		
		%__________________________
		%Estudiante
		Quintero Celis Samuel David & 
		%Disciplinario
		Es un estudiante que se le hacen reiterados llamados de atención, por su comportamiento en clase y el trato con sus compañeros, uso de lenguaje soez para referirse a sus pares, el estudiante tiene compromiso por comportamiento.  & 
		%Académico
		El estudiante pierde Microcomunidad.
		El estudiante tiene con pérdida: 3 desempeños de Microcomunidad, 2 desempeños de Actitudinal de Padres, 1 desempeños de Español, 1 desempeños de Química, 1 desempeños de Física, 1 desempeño de Trigonometría y 1 desempeño de Ética. & 
		%¿Qué ha hecho?
		Seguimiento continúo académico y convivencial del estudiante, seguimiento de huellas por plataforma y agenda. Diálogo constante con la estudiante, aconsejando y acompañando todos los días. Citación a acudientes.& 
		%¿Qué propone hacer?
		Continuar con el seguimiento constante, seguir concientizando a los estudiantes sobre el cumplimiento de las normas del colegio, fomentar aún más el compañerismo, seguir recalcando en los estudiantes las buenas costumbres y el buen trato entre compañeros y docentes. Seguir motivando a ser mejores personas día con día.\\
		\hline
		
		%__________________________
		%Estudiante
		Riaño Higuera Paula Sofia & 
		%Disciplinario
		La estudiante debe firmar compromiso por retardos.
		Es una estudiante que sigue las indicaciones del docente, evita cometer faltas de cualquier tipo, presenta una buena conducta en el aula de clases. & 
		%Académico
		La estudiante tiene con pérdida: 1 desempeño de Microcomunidad, 1 desempeño de Actitudinal de Padres, 1 desempeño de Reading and Writing, 2 desempeños de Química y 3 desempeños de Trigonometría. & 
		%¿Qué ha hecho?
		Seguimiento continúo académico y convivencial de la estudiante, seguimiento de huellas por plataforma y agenda. Diálogo constante con la estudiante, aconsejando y acompañando todos los días. & 
		%¿Qué propone hacer?
		Continuar con el seguimiento constante, seguir concientizando a los estudiantes sobre el cumplimiento de las normas del colegio, fomentar aún más el compañerismo, seguir recalcando en los estudiantes las buenas costumbres y el buen trato entre compañeros y docentes. Seguir motivando a ser mejores personas día con día.\\
		\hline
		
		%__________________________
		%Estudiante
		Rodriguez Gonzalez Samuel Francisco & 
		%Disciplinario
		Es un estudiante que sigue las indicaciones del docente, evita cometer faltas de cualquier tipo, presenta una buena conducta en el aula de clases. & 
		%Académico
		El estudiante tiene con pérdida: 2 desempeños de Actitudinal de Padres. & 
		%¿Qué ha hecho?
		Seguimiento continúo académico y convivencial de la estudiante, seguimiento de huellas por plataforma y agenda. Diálogo constante con la estudiante, aconsejando y acompañando todos los días. & 
		%¿Qué propone hacer?
		Continuar con el seguimiento constante, seguir concientizando a los estudiantes sobre el cumplimiento de las normas del colegio, fomentar aún más el compañerismo, seguir recalcando en los estudiantes las buenas costumbres y el buen trato entre compañeros y docentes. Seguir motivando a ser mejores personas día con día.\\
		\hline
		
		%__________________________
		%Estudiante
		Romero Díaz Mariana & 
		%Disciplinario
		Es una estudiante que sigue las indicaciones del docente, evita cometer faltas de cualquier tipo, presenta una buena conducta en el aula de clases. & 
		%Académico
		La estudiante tiene con pérdida: 1 desempeño de Actitudinal de Padres, 1 desempeño de Química, 1 desempeño de Física y 2 desempeños de Trigonometría. & 
		%¿Qué ha hecho?
		Seguimiento continúo académico y convivencial de la estudiante, seguimiento de huellas por plataforma y agenda. Diálogo constante con la estudiante, aconsejando y acompañando todos los días. & 
		%¿Qué propone hacer?
		Continuar con el seguimiento constante, seguir concientizando a los estudiantes sobre el cumplimiento de las normas del colegio, fomentar aún más el compañerismo, seguir recalcando en los estudiantes las buenas costumbres y el buen trato entre compañeros y docentes. Seguir motivando a ser mejores personas día con día.\\
		\hline
		
		%__________________________
	\end{longtable}
	
	\begin{flushleft}
		A continuación se presentan los estudiantes que deben firmar compromiso:\\
	\end{flushleft}
	\vspace{0.5cm}
	\begin{longtable}{|p{6cm}|p{5cm}|}
		\hline
		\textbf{Estudiante} & \textbf{Compromiso} \\
		\hline\hline
		Matiz Camargo Angel Santiago & Compromiso por retardos.  \\
		\hline
		Riaño Higuera Paula Sofía & Compromiso por retardos.  \\
		\hline
		Quintero Celis Samuel David & Compromiso de convivencia.
		Compromiso académico  \\
		\hline
		Moreno Buitrago Samuel & Compromiso de convivencia.  \\
		\hline
		Acevedo Vanegas Isabella & Compromiso académico.  \\
		\hline
		Chavarro Reyes Lorieth Sofia & Compromiso académico.  \\
		\hline
	\end{longtable}
	
	
\end{document}\\
